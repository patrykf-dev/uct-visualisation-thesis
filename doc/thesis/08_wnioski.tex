\documentclass[uct_visualisation_thesis.tex]{subfiles}
Praca nad projektem powiązanym z tematyką sztucznej inteligencji pozwoliła autorom zrozumieć dokładną zasadę działania algorytmu UCT i sprawiła, że zgłębili oni wiedzę na temat jednego z popularniejszych algorytmów dotyczących symulacji obiecujących ruchów komputera w grach logicznych.\\

Proces implementacji wizualizacji drzewa stanów uświadomił autorom, jak bardzo wydajne może być zastosowanie karty graficznej, w celu wykonywania sprawnych obliczeń związanych z grafiką. Utwierdził ich też w przekonaniu, jak dużo możliwości daje korzystanie z biblioteki \textit{OpenGL}.\\

Dodatkowo podczas pracy autorzy nauczyli się efektywnego łączenia komponentów z~różnych bibliotek w celu stworzenia jednej spójnej aplikacji, a także sprawnego korzystania z~zewnętrznych dokumentacji. Zaobserwowano również, że biblioteki przyciągające programistów swoją łatwością w obsłudze nie zawsze są najlepszym wyborem. Za przykład może posłużyć używana biblioteka \textit{PyGame} do graficznej reprezentacji gier. Ostatecznie zrezygnowano z niej, ponieważ nie była ona kompatybilna z innymi komponentami, między innymi z modułem \textit{PyQt}, odpowiedzialnym za tworzenie interfejsu graficznego.\\

Zmierzono się również z problemem dotyczącym rysowania drzew nieprzecinających się krawędziami, w sposób możliwie zwięzły. Okazało się, że znaleziony w opracowaniach naukowych ulepszony algorytm Walkera nie wydaje się bardzo popularny, mimo swojego czasu obliczeń rzędu $\Theta (n)$. Pozwoliło to uświadomić autorom, jak wiele wydajnych, a jednocześnie mało znanych rozwiązań można znaleźć we wszelkiego rodzaju źródłach naukowych.\\

Podczas pracy zrozumiano, że trendy panujące na rynku informatycznym dotyczące języków programowania nie zawsze współgrają z ich wydajnością. Przykładowo, \textit{Python} ze względu na swoje dynamiczne typowanie nie jest najlepszym wyborem do programów wymagających wykonywania bardzo dużej ilości obliczeń i ustępuje szybkością starszym językom programowania, takim jak na przykład \textit{C}. Ze względu na swoją nieskomplikowaną składnię jest on natomiast dobrym wyborem do zadań, które wymagają napisania kodu w szybki sposób, lecz niekoniecznie optymalnie działającego.\\
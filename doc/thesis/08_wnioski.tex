\documentclass[uct_visualisation_thesis.tex]{subfiles}
Praca nad projektem powiązanym z tematem sztucznej inteligencji pozwoliła zrozumieć nam dokładną zasadę działania algorytmu UCT i sprawiła, że zgłębiliśmy naszą wiedzę na temat jednego z popularniejszych algorytmów dotyczących symulacji obiecujących ruchów komputera w grach logicznych.\\

Proces implementacji wizualizacji drzewa stanów uświadomił nam, jak bardzo wydajne może być zastosowanie karty graficznej, w celu wykonywania sprawnych obliczeń związanych z grafiką. Utwierdził nas on też w przekonaniu, jak dużo możliwości daje korzystanie z biblioteki \textit{OpenGL}.\\

Dodatkowo, w naszej pracy nauczyliśmy się efektywnego łączenia komponentów z różnych bibliotek w celu stworzenia jednej spójnej aplikacji. Nauczyło nas to także sprawnego korzystania z zewnętrznych dokumentacji. Zauważyliśmy również, że biblioteki przyciągające programistów swoją łatwością w obsłudze nie zawsze są najlepszym wyborem. Za przykład może posłużyć używana przez nas biblioteka \textit{PyGame} do graficznej reprezentacji gier. Ostatecznie zrezygnowaliśmy z niej, ponieważ nie była ona kompatybilna z innymi komponentami, między innymi z modułem \textit{PyQt}, odpowiedzialnym za tworzenie interfejsu graficznego.\\

Podczas pracy zrozumieliśmy, że trendy panujące na rynku informatycznym dotyczące języków programowania nie zawsze współgrają z ich wydajnością. Przykładowo, Python ze względu na bycie językiem interpretowanym nie jest najlepszym wyborem do programów wymagających wykonywania bardzo dużej ilości obliczeń i ustępuje szybkością starszym językom programowania, tak jak na przykład C. Jest on natomiast dobrym wyborem do zadań, które wymagają napisania kodu szybkiego, lecz niekoniecznie optymalnie działającego.\\

Zmierzyliśmy się również z problemem dotyczącym rysowania drzew nieprzecinających się krawędziami, w sposób możliwie zwięzły. Okazało się, że znaleziony przez nas w opracowaniach naukowych ulepszony algorytm Walkera nie wydaje się bardzo popularny, mimo swojego czasu obliczeń rzędu $\Theta (n)$. Uświadomiło nas to, jak wiele wydajnych, a jednocześnie mało znanych rozwiązań można znaleźć w opracowaniach naukowych.\\
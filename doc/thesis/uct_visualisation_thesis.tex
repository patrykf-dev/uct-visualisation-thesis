\documentclass[a4paper,11pt,twoside]{report}
% KOMPILOWAĆ ZA POMOCĄ pdfLaTeXa, PRZEZ XeLaTeXa MOŻE NIE BYĆ POLSKICH ZNAKÓW

\usepackage{subfiles}
% -------------- Kodowanie znaków, język polski -------------

%\usepackage[utf8]{inputenc}
%\usepackage[MeX]{polski}
%\usepackage[T1]{fontenc}
\usepackage[OT4]{fontenc}
\usepackage[english,polish]{babel}


\usepackage{amsmath, amsfonts, amsthm, latexsym} % głównie symbole matematyczne, środowiska twierdzeń

\usepackage[final]{pdfpages} % inputowanie pdfa
%\usepackage[backend=bibtex, style=verbose-trad2]{biblatex}


% ---------------- Wczytywanie grafik ------------------

\usepackage{graphicx}
\graphicspath{{grafiki/}{../grafiki/}}


% ---------------- Tabele ------------------

\usepackage{float} % for "H" placement specifier

% ---------------------- Hiperłącza --------------------------

\usepackage{hyperref}
\PassOptionsToPackage{hyphens}{url}\usepackage{hyperref}

% ---------------- Marginesy, akapity, interlinia ------------------


\usepackage[inner=20mm, outer=20mm, bindingoffset=10mm, top=25mm, bottom=25mm]{geometry}


\linespread{1.5}
\allowdisplaybreaks

\usepackage{indentfirst} % opcjonalnie; pierwszy akapit z wcięciem
\setlength{\parindent}{5mm}


% ---------------- Formatowanie pseudokodu  ------------------

\usepackage{xcolor}
\usepackage{listings}
\definecolor{codegreen}{rgb}{0,0.7,0}
\definecolor{codegray}{rgb}{0.5,0.5,0.5}
\definecolor{codepurple}{rgb}{0.58,0,0.82}
\definecolor{backcolour}{rgb}{0.95,0.95,0.92}
\lstdefinestyle{mystyle}{
	language=Python,
	deletekeywords={from},
	backgroundcolor=\color{backcolour},   
	commentstyle=\color{codegreen},
	keywordstyle=\color{magenta},
	numberstyle=\tiny\color{codegray},
	stringstyle=\color{codepurple},
	basicstyle=\ttfamily\footnotesize,
	breakatwhitespace=false,         
	breaklines=true,                 
	captionpos=b,                    
	keepspaces=true,                 
	numbers=left,                    
	numbersep=5pt,                  
	showspaces=false,                
	showstringspaces=false,
	showtabs=false,                  
	tabsize=4
}
\lstdefinestyle{mystylenonumbers}{
	language=Python,
	deletekeywords={from},
	backgroundcolor=\color{backcolour},   
	commentstyle=\color{codegreen},
	keywordstyle=\color{magenta},
	numberstyle=\tiny\color{codegray},
	stringstyle=\color{codepurple},
	basicstyle=\ttfamily\footnotesize,
	breakatwhitespace=false,         
	breaklines=true,                 
	captionpos=b,                    
	keepspaces=true,                 
	numbers=none,                    
	numbersep=5pt,                  
	showspaces=false,                
	showstringspaces=false,
	showtabs=false,                  
	tabsize=4
}

%--------------------------- ŻYWA PAGINA ------------------------

\usepackage{fancyhdr}
\pagestyle{fancy}
\fancyhf{}
% numery stron: lewa do lewego, prawa do prawego 
\fancyfoot[LE,RO]{\thepage} 
% prawa pagina: zawartość \rightmark do lewego, wewnętrznego (marginesu) 
\fancyhead[LO]{\sc \nouppercase{\rightmark}}
% lewa pagina: zawartość \leftmark do prawego, wewnętrznego (marginesu) 
\fancyhead[RE]{\sc \leftmark}

\renewcommand{\chaptermark}[1]{
\markboth{\thechapter.\ #1}{}}

% kreski oddzielające paginy (górną i dolną):
\renewcommand{\headrulewidth}{0 pt} % 0 - nie ma, 0.5 - jest linia


\fancypagestyle{plain}{% to definiuje wygląd pierwszej strony nowego rozdziału - obecnie tylko numeracja
  \fancyhf{}%
  \fancyfoot[LE,RO]{\thepage}%
  
  \renewcommand{\headrulewidth}{0pt}% Line at the header invisible
  \renewcommand{\footrulewidth}{0.0pt}
}

% ---------------- Nagłówki rozdziałów ---------------------

\usepackage{titlesec}
\titleformat{\chapter}%[display]
  {\normalfont\Large \bfseries}
  {\thechapter.}{1ex}{\Large}

\titleformat{\section}
  {\normalfont\large\bfseries}
  {\thesection.}{1ex}{}
\titlespacing{\section}{0pt}{30pt}{20pt} 
%\titlespacing{\co}{akapit}{ile przed}{ile po} 
    
\titleformat{\subsection}
  {\normalfont \bfseries}
  {\thesubsection.}{1ex}{}


% ----------------------- Spis treści ---------------------------
\def\cleardoublepage{\clearpage\if@twoside
\ifodd\c@page\else\hbox{}\thispagestyle{empty}\newpage
\if@twocolumn\hbox{}\newpage\fi\fi\fi}


% kropki dla chapterów
\usepackage{etoolbox}
\makeatletter
\patchcmd{\l@chapter}
  {\hfil}
  {\leaders\hbox{\normalfont$\m@th\mkern \@dotsep mu\hbox{.}\mkern \@dotsep mu$}\hfill}
  {}{}
\makeatother

\usepackage{titletoc}
\makeatletter
\titlecontents{chapter}% <section-type>
  [0pt]% <left>
  {}% <above-code>
  {\bfseries \thecontentslabel.\quad}% <numbered-entry-format>
  {\bfseries}% <numberless-entry-format>
  {\bfseries\leaders\hbox{\normalfont$\m@th\mkern \@dotsep mu\hbox{.}\mkern \@dotsep mu$}\hfill\contentspage}% <filler-page-format>

\titlecontents{section}
  [1em]
  {}
  {\thecontentslabel.\quad}
  {}
  {\leaders\hbox{\normalfont$\m@th\mkern \@dotsep mu\hbox{.}\mkern \@dotsep mu$}\hfill\contentspage}

\titlecontents{subsection}
  [2em]
  {}
  {\thecontentslabel.\quad}
  {}
  {\leaders\hbox{\normalfont$\m@th\mkern \@dotsep mu\hbox{.}\mkern \@dotsep mu$}\hfill\contentspage}
\makeatother



% ---------------------- Spisy tabel i obrazków ----------------------

\renewcommand*{\thetable}{\arabic{chapter}.\arabic{table}}
\renewcommand*{\thefigure}{\arabic{chapter}.\arabic{figure}}
\usepackage{caption}
\captionsetup[table]{name=Tabela}
%\let\c@table\c@figure % jeśli włączone, numeruje tabele i obrazki razem


% --------------------- Definicje, twierdzenia etc. ---------------


\makeatletter
\newtheoremstyle{definition}%    % Name
{3ex}%                          % Space above
{3ex}%                          % Space below
{\upshape}%                      % Body font
{}%                              % Indent amount
{\bfseries}%                     % Theorem head font
{.}%                             % Punctuation after theorem head
{.5em}%                            % Space after theorem head, ' ', or \newline
{\thmname{#1}\thmnumber{ #2}\thmnote{ (#3)}}%  % Theorem head spec (can be left empty, meaning `normal')
\makeatother

% ----------------------------- POLSKI --------------------------------

\theoremstyle{definition}
\newtheorem{theorem}{Twierdzenie}[chapter]
\newtheorem{lemma}[theorem]{Lemat}
\newtheorem{example}[theorem]{Przykład}
\newtheorem{proposition}[theorem]{Stwierdzenie}
\newtheorem{corollary}[theorem]{Wniosek}
\newtheorem{definition}[theorem]{Definicja}
\newtheorem{remark}[theorem]{Uwaga}

% -------------------------- POCZĄTEK --------------------------


% --------------------- Ustawienia użytkownika ------------------

\newcommand{\tytul}{Wizualizacja drzewa stanów algorytmu UCT}
\newcommand{\tytulen}{Visualization of UCT trees}
\newcommand{\type}{inżyniers}
\newcommand{\supervisor}{mgr inż. Jan Karwowski }



\begin{document}
\sloppy

\includepdf[pages=-]{titlepage}


% ------------------ STRONA Z PODPISAMI AUTORA/AUTORÓW I PROMOTORA ------------------


\thispagestyle{empty}\newpage
\null

\vfill

\begin{center}
\begin{tabular}[t]{ccc}

............................................. & \hspace*{100pt} & .............................................\\
podpis promotora & \hspace*{100pt} & podpis autora


\end{tabular}
\end{center}



% ---------------------------- ABSTRAKTY -----------------------------
% W PRACY PO POLSKU, NAPIERW STRESZCZENIE PL, POTEM ABSTRACT EN

{
\begin{abstract}

\begin{center}
\tytul
\end{center}

Tematem pracy inżynierskiej jest implementacja projektu wizualizacji drzew stanów algorytmu z dziedziny sztucznej inteligencji -- Upper Confidence Bound Applied to Trees (UCT). Prezentowane rozwiązanie wykorzystuje wspomniany algorytm przy podejmowaniu decyzji podczas rozgrywki w dwie przykładowe gry planszowe - szachy i mankalę.\\

Kluczową funkcjonalnością prezentowanego systemu jest przejrzysta wizualizacja ukazująca kolejne etapy rozrastania się drzewa stanów. Aplikacja jest wygodnym narzędziem do analizy działania algorytmu w czasie rzeczywistym. Moduł odpowiedzialny za wizualizację, będący najistotniejszym, zawiera implementację usprawnionej wersji algorytmu Walkera. Opisane są również moduły odpowiedzialne za logikę zaimplementowanych gier i sposób ich ewaluacji przez algorytm, implementację algorytmu UCT, a także schematy serializacji generowanych drzew do plików binarnych i tekstowych. Opis powyższych komponentów i funkcjonalności systemu wzbogacony jest o liczne diagramy ilustrujące dane zagadnienie. Przedstawiony jest również obszerny opis interfejsu użytkownika ukazujący najistotniejsze okna aplikacji i przeprowadzający czytelnika przez funkcjonalności udostępniane przez system i sposób ich użycia. Ukazane w dokumencie instrukcje instalacji prowadzą użytkownika przez proces instalacji programu na systemach operacyjnych Windows oraz Linux. Finalnie, opisane zostały wnioski autorów powstałe podczas pracy nad projektem dotyczące wykorzystanych technologii, takie jak przemyślenia o tym, czy język Python sprawdził się pod względem efektywności jako język programowania użyty w aplikacji.\\

\noindent \textbf{Słowa kluczowe:} UCT, MCTS, wizualizacja, sztuczna inteligencja, gry logiczne
\end{abstract}
}

\null\thispagestyle{empty}\newpage

{\selectlanguage{english}
\begin{abstract}

\begin{center}
\tytulen
\end{center}

The topic of this engineering diploma project is to create a system that can visualise state trees of the artificial intelligence algorithm -- Upper Confidence Bound Applied to Trees (UCT). The presented solution uses the algorithm to make decisions while playing two example board games -- chess and mancala. \\

The essential functionality of the presented system is a precise visualisation showing the subsequent stages of tree growth. The application is a convenient tool for analyzing the algorithm's operations in real-time. The module responsible for visualization, which is the most important, contains an implementation of the Improved Walker's Algorithm. The work also contains descriptions of modules responsible for games' logic and their evaluation methods, implementation of the UCT algorithm, and the serialization schemes of generated trees to binary and text files. The components mentioned above and system functionalities descriptions are enriched with numerous diagrams illustrating a specific issue. A comprehensive description of the graphical user interface is presented as well, showing the most significant application windows and guiding the reader through the functionalities provided by the system and their methods of usage. The system installation manuals lead the user through the installation process on Windows and Linux operating systems. Finally, there are listed authors' conclusions drawn from the work on the project relating to the used technologies, including thoughts about the effectiveness of Python as application's programming language. \\

\noindent \textbf{Keywords:} UCT, MCTS, visualisation, artificial intelligence, logic games
\end{abstract}
}


% --------------------- OŚWIADCZENIE -----------------------------------------


\null\thispagestyle{empty}\newpage

\null \hfill Warszawa, dnia ..................\\

\par\vspace{5cm}

\begin{center}
Oświadczenie
\end{center}

\indent Oświadczam, że moją część pracy inżynierskiej (zgodnie z podziałem zadań opisanym w pkt. \ref{sec:workdivision}) pod tytułem ,,\tytul '', której promotorem jest \supervisor wykonałem samodzielnie, co poświadczam własnoręcznym podpisem.
\vspace{2cm}

\begin{flushright}
  \begin{minipage}{50mm}
    \begin{center}
      ..............................................

    \end{center}
  \end{minipage}
\end{flushright}

\thispagestyle{empty}
\newpage

\null\thispagestyle{empty}\newpage


% ------------------- 4. Spis treści ---------------------
\pagenumbering{gobble}
\tableofcontents
\thispagestyle{empty}

\newpage % JEŻELI SPIS TREŚCI MA PARZYSTĄ LICZBĘ STRON, ZAKOMENTOWAĆ
% ALBO JAK KTOŚ WOLI WTEDY DWIE STRONY ODSTĘPU, DODAĆ \null\newpage

% -------------- 5. ZASADNICZA CZĘŚĆ PRACY --------------------
\null\thispagestyle{empty}\newpage
\pagestyle{fancy}
\pagenumbering{arabic}
\setcounter{page}{11} % JEŻELI Z POWODU DUŻEJ ILOŚCI STRON W SPISIE TREŚCI SIĘ NIE ZGADZA, TRZEBA ZMODYFIKOWAĆ RĘCZNIE

\chapter*{Wykaz najważniejszych oznaczeń i skrótów}
\subfile{00_slowniczek}

\chapter{Wstęp}
\subfile{000_wstep}

\chapter{Założenia projektowe} \label{chap:zal_pr}
\subfile{01_wstep}

\chapter{Teoria} \label{chap:teoria}
\subfile{02_teoria}

\chapter{Implementacja} \label{chap:impl}
\subfile{03_implementacja}

\chapter{Główne komponenty aplikacji} \label{chap:glowne_komp}
\subfile{04_komponenty}

\chapter{Interfejs użytkownika} \label{chap:ui}
\subfile{05_interfejs}

\chapter{Instrukcja instalacji} \label{chap:insr_instalacji}
\subfile{06_instrukcje}

\chapter{Podsumowanie i ocena} \label{chap:podsumowanie}
\subfile{07_podsumowanie}

\chapter{Wnioski} \label{chap:wnioski}
\subfile{08_wnioski}

% -------------------- 6. Bibliografia -----------------------
% Bibliografia leksykograficznie wg nazwisk autorów

\begin{thebibliography}{20}%jak ktoś ma więcej książek, to niech wpisze większą liczbę
% \bibitem[numerek]{referencja} Autor, \emph{Tytuł}, Wydawnictwo, rok, strony
% cytowanie: \cite{referencja1, referencja 2,...}
\bibitem[1]{banditbased} Levente Kocsis, Csaba Szepesvári, \emph{Bandit based Monte-Carlo Planning}, European Conference on Machine Learning, Berlin, Germany, September 18--22, 2006.
\bibitem[2]{mctsanalysis} Steven James, George Konidaris, Benjamin Rosman, \emph{An Analysis of Monte Carlo Tree Search}, University of the Witwatersrand, Johannesburg, South Africa.
\bibitem[3]{treelayout} K. Marriott, \emph{NP-Completeness of Minimal Width Unordered Tree Layout}, Journal of Graph Algorithms and Applications, vol. 8, no. 3, pp. 295--312 (2004).
\bibitem[4]{impwalkers} Christop Buchheim, Michael Jünger, Sebastian Leipert, \emph{Improving Walker's Algorithm to Run in Linear Time}, Universität zu Köln, Institut für Informatik, 2002.
\bibitem[5]{mctspatrolling} Jan Karwowski, Jacek Mańdziuk, \emph{A Monte Carlo Tree Search approach to finding efficient patrolling schemes on graphs}, European Journal of Operational Research, vol. 277, no. 1, pp. 255-268 (2019).
\end{thebibliography}

\thispagestyle{empty}
\pagenumbering{gobble}

% ----- 8. Spis rysunków - jeśli nie ma, zakomentować --------
\listoffigures
\thispagestyle{empty}


% ------------ 9. Spis tabel - jak wyżej ------------------
\renewcommand{\listtablename}{Spis tabel}
\listoftables
\thispagestyle{empty}


% 10. Spis załączników - jak nie ma załączników, to zakomentować lub usunąć

\chapter*{Spis załączników}
\begin{enumerate}
\item Płyta CD zawierająca:
\begin{itemize}
	\item dokument z treścią pracy dyplomowej,
	\item streszczenie w języku polskim,
	\item streszczenie w języku angielskim,
	\item kod programu,
	\item przenośną wersję aplikacji,
	\item dokumentację aplikacji.
\end{itemize}
\end{enumerate}
\thispagestyle{empty}


\end{document}

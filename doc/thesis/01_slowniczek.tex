\documentclass[uct_visualisation_thesis.tex]{subfiles}

\begin{itemize}
	\item \textbf{MCTS} (Monte Carlo Tree Search) - heurystyka podejmowania decyzji w pewnych zadaniach sztucznej inteligencji, np. ruchów w grach. Najczęściej MCTS opiera się na wariancie metody UCT.
	\item \textbf{UCT} (Upper Confidence Bound Applied to Trees) - algorytm przeszukujący drzewo stanów rozgrywki w poszukiwaniu najbardziej opłacalnych ruchów. Algorytm stara się zachować równowagę między eksploatacją ruchów po ruchach o wysokiej średniej wygranej a eksploracją tych mało sprawdzonych.
	\item \textbf{Podstawowa wizualizacja} - możliwość wyświetlenia całego drzewa stanów. W podstawowej wizualizacji wygląd wierzchołków i krawędzi jest nieistotny.
	\item \textbf{Zaawansowana wizualizacja} - podstawowa wizualizacja wzbogacona o możliwość analizowania statystyk poszczególnych wierzchołków. Kolor wierzchołków będzie reprezentował aktualnego gracza, a kolor krawędzi częstość odwiedzin danego węzła. Możliwe powinno być też przewijanie, przybliżanie oraz oddalanie podglądu drzewa.
	\item \textbf{CSV} - plik w formacie .csv (ang. \textit{comma-separated values}) służący do przechowywania danych w plikach tekstowych, gdzie separatorem jest przecinek.
\end{itemize}
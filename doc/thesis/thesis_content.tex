\documentclass[uct_visualisation_thesis.tex]{subfiles}

% ------------------- 1. SŁOWNICZEK ---------------------
\chapter{Wykaz najważniejszych oznaczeń i skrótów}
\begin{itemize}
	\item \textbf{MCTS} (Monte Carlo Tree Search) - heurystyka podejmowania decyzji w pewnych zadaniach sztucznej inteligencji, np. ruchów w grach. Najczęściej MCTS opiera się na wariancie metody UCT.
	\item \textbf{UCT} (Upper Confidence Bound Applied to Trees) - algorytm przeszukujący drzewo stanów rozgrywki w poszukiwaniu najbardziej opłacalnych ruchów. Algorytm stara się zachować równowagę między eksploatacją ruchów po ruchach o wysokiej średniej wygranej a eksploracją tych mało sprawdzonych.
	\item \textbf{Podstawowa wizualizacja} - możliwość wyświetlenia całego drzewa stanów. W podstawowej wizualizacji wygląd wierzchołków i krawędzi jest nieistotny.
	\item \textbf{Zaawansowana wizualizacja} - podstawowa wizualizacja wzbogacona o możliwość analizowania statystyk poszczególnych wierzchołków. Kolor wierzchołków będzie reprezentował aktualnego gracza, a kolor krawędzi częstość odwiedzin danego węzła. Możliwe powinno być też przewijanie, przybliżanie oraz oddalanie podglądu drzewa.
	\item \textbf{CSV} - plik w formacie .csv (ang. \textit{comma-separated values}) służący do przechowywania danych w plikach tekstowych, gdzie separatorem jest przecinek.
\end{itemize}

% ------------------- 2. WSTĘP ---------------------
\chapter{Wstęp i cel pracy}
\section{Opis problemu klienta}

\section{Cel biznesowy}
Algorytm UCT, będący usprawnieniem MCTS, jest powszechnie stosowanym algorytmem w sztucznej inteligencji. Jest metodą analizującą obiecujące ruchy na podstawie generowanego drzewa, która równoważy eksploatację najbardziej korzystnych z eksploracją mniej korzystnych decyzji. Każdemu wierzchołkowi drzewa odpowiada pewien stan rozgrywki, z którego algorytm rozgrywa losowe symulacje, rozszerzając potem drzewo o kolejne możliwe stany. Sposób, w jaki rozrasta się opisywane drzewo, jest kluczowy dla podejmowania przez algorytm jak najlepszych decyzji.\\

Celem projektu jest stworzenie aplikacji pozwalającej na wizualizację drzew algorytmu UCT. Aplikacja będzie pozwalała na wizualizowanie drzew generowanych podczas rozgrywania dwóch przykładowych gier (pozwalając przetestować rozwiązanie). Aplikacja powinna pozwalać na wizualizację drzew, ich sekwencji i róznic między kolejnymi drzewami w sekwencji. Powinna być możliwość płynnego przybliżania/oddalania i przewijania wizualizacji oraz zapisu aktualnego stanu do pliku graficznego - wszystko, aby klient mógł wygodnie korzystać z naszego programu.\\

Taki produkt pozwoliłby zrozumieć klientowi ideę i sposób działania algorytmu UCT.
\section{Założenia projektowe}
\subsection{Założenia funkcjonalne}
\subsection{Założenia niefunkcjonalne}


% ------------------- 3. TEORIA ---------------------
\chapter{Teoria}
\section{Algorytmy MCTS}
\subsection{Opis grupy algorytmów}
\textbf{MCTS} (Monte Carlo Tree Search) - heurystyka podejmowania decyzji w pewnych zadaniach sztucznej inteligencji, np. ruchów w grach. Najczęściej MCTS opiera się na jakimś wariancie metody UCT.

\section{Algorytm UCT}
\textbf{UCT} (Upper Confidence Bound Applied to Trees) - algorytm przeszukujący drzewo stanów rozgrywki w poszukiwaniu najbardziej opłacalnych ruchów. Algorytm stara się zachować równowagę między eksploatacją ruchów po ruchach o wysokiej średniej wygranej a eksploracją tych mało sprawdzonych.
\subsection{Opis algorytmu}
\subsection{Dodatkowe założenia}
\section{Algorytm wizualizacji drzewa}
\subsection{Określenie problematyki}
\subsection{Usprawniony algorytm Walkera}


% ------------------- 4. IMPLEMENTACJA ---------------------
\chapter{Implementacja}
\section{Wykorzystane technologie}
\section{Architektura i działanie systemu}
\subsection{Moduły}
\subsection{Główne komponenty aplikacji}
\subsection{Interfejs użytkownika}


% ------------------- 5. INSTRUKCJE ---------------------
\chapter{Instrukcje}
\section{Instrukcja instalacji}
\section{Instrukcja użytkownika}


% ------------------- 6. OCENA I PODSUMOWANIE---------------------
\chapter{Podsumowanie i ocena}
\section{Uzyskane efekty}
\section{Kontynuacja pracy}
\section{Wydajność}
\section{Testy akceptacyjne}


% ------------------- 6. WNIOSKI ---------------------
\chapter{Wnioski}

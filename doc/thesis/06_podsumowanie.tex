\documentclass[uct_visualisation_thesis.tex]{subfiles}

\section{Uzyskane efekty}
\section{Kontynuacja pracy}
Aplikacja spełnia wszystkie wymagania określone w opisie tematu pracy, jednak w przyszłości mogą zostać do niej dodane usprawnienia. W celu dalszej poprawy jakości i zwiększenia liczby możliwych opcji, obecny produkt można rozszerzyć o następujące funkcjonalności:
\begin{enumerate}
	\item \textbf{Dodanie kolejnych gier} - obecna architektura projektu w dużym stopniu umożliwia rozszerzenie aplikacji o kolejne gry logiczne, które spełniają założenia algorytmu UCT.
	\item \textbf{Usprawnienie i zrównoleglenie symulacji szachów} - symulacja ruchu szachowego przez komputer zajmuje dużo więcej czasu w przypadku szachów niż mankali. W takim wypadku można pokusić się o zrównoleglenie pewnych czynności szachowych, takie jak znajdowanie wszystkich możliwych dozwolonych ruchów, które zajmują najwięcej czasu. Przy przeprowadzonych próbach okazało się, że znajdowanie , jednakże można by poszukać sposobu implementacji wielowątkowości, który sprawiałby, że symulacja ruchu szachowego przez komputer działała szybciej i wydajniej, jednocześnie nie powodując przy tym dodatkowych błędów.
	\item \textbf{Dodanie księgi otwarć dla szachów} - algorytm wyznaczający ruch komputera w rozgrywce szachowej 
	\item \textbf{Lepsza ewaluacja wartości figur w szachach} - sposób wartościowania figur jest miejscem, w którym można znacznie rozwinąć potencjał decyzji podejmowanych przez algorytm UCT. Na ten moment każda figura ma arbitralnie ustaloną wartość, bez względu na swoje położenie na szachownicy. Tymczasem, wartość danej figury może zależeć od wielu czynników, między innymi od pozycji, w której się znajduje. Dla przykładu, koń w centrum planszy będzie miał większy potenciał od takiego, który umieszczony jest w rogu. Tak samo pion, który jest bliżej ostatniego rzędu pól, jest bardziej wartościowy od piona na swojej początkowej pozycji, a pion zdublowany prawdopodobnie jeszcze mniej. Liczba konfiguracji i czynników wpływających na potencjał figury w danym miejscu na szachownicy wykracza poza temat tej pracy, jednak potencjalne rozwinięcie lepiej przemyślanej ewaluacji figur wpłynęło by korzystnie na zdolność algorytmu do podejmowania bardziej obiecujących ruchów.
	\item \textbf{Inteligentne przydzielanie pamięci dla drzew w sekwencji} - podczas przełączania się pomiędzy kolejnymi drzewami w sekwencji, za każdym razem następuje wczytanie do pamięci wartości i pozycji wszystkich węzłów danego drzewa. Ocznacza to, że jeśli użytkownik analizuje dwa kolejne drzewa i przełącza między nimi na zmianę, za każdym razem te same drzewa są wczytywane do pamięci na nowo. Usprawnieniem tego procesu byłoby przechowywanie wcześniej już wczytanych drzew w pamięci podręcznej, jednocześnie automatycznie kontrolując ilość wykorzystanych zasobów. Co więcej, program mógłby wczytywać od razu pewną liczbę kolejnych drzew z wyprzedzeniem. Takie operacje zaoszczędziłyby czas użytkownika, zwłaszcza podczas wczytywania drzew z dużą ilością węzłów.
\end{enumerate}
Co więcej, w celu usprawnienia modułów, 
 Przepisanie modułów odpowiedzialnych za obliczenia do C/C++
\section{Wydajność}
\section{Testy akceptacyjne}
Testy akceptacyjne zostały przeprowadzone w celu sprawdzenia, czy aplikacja spełnia założenia opisane w dokumentacji wymagań projektu. Test \ref{tab:test1} konfrontuje założenia modułu \textit{Gry}, test \ref{tab:test2} – modułu \textit{Serializacja}, a pozostałe testy weryfikują założenia modułu \textit{Wizualizacja}.\\

Testy akceptacyjne zostały wykonane na komputerze:

\begin{itemize}
	\item z zainstalowanym systemem operacyjnym \textit{Windows 10 Education N},
	\item z zainstalowanym interpreterem języka \textit{Python 3.7.2} i biblioteką \textit{PyQt5},
	\item wyposażonym w procesor \textit{Intel Core i7-8700k @3.70 GHz},
	\item wyposażonym w kartę graficzną \textit{NVIDIA GeForce GTX 1060 6GB},
	\item wyposażonym w 32GB pamięci RAM.
\end{itemize}

Pierwszy z przeprowadzonych testów dotyczy funkcjonalności modułu \textit{Gry}, a wynik opisany został w tabeli \ref{tab:test1}.

\begin{table}[H]
	\caption{Raport z pierwszego testu}
	\label{tab:test1}
	\centering
	\begin{tabular}{|c|p{0.6\textwidth}|}
		\hline
		Testowane wymaganie & \textit{Użytkownik będzie mógł wybrać jedną z dwóch przykładowych gier, a do wyboru będzie miał trzy tryby rozgrywki.} \\ \hline
		Kroki testowe & \begin{enumerate} \item Z menu głównego aplikacji wybierz opcję \textit{Chess}. \item Z menu głównego aplikacji wybierz opcję \textit{Player vs player} i sprawdź tryb rozgrywki dla dwóch graczy. \item Z menu głównego aplikacji wybierz opcję \textit{Player vs PC} dla różnych ustawień algorytmu UCT. \item Z menu głównego aplikacji wybierz opcję \textit{PC vs PC} dla różnych ustawień algorytmu UCT. \item Z menu głównego aplikacji wybierz \textit{Mancala} i powtórz kroki 2–5. \end{enumerate} \\ \hline
		Wynik & Pozytywny. \\ \hline
	\end{tabular}
\end{table}

Drugi z przeprowadzonych testów dotyczy funkcjonalności modułu \textit{Serializacja}, a wynik opisany został w tabeli \ref{tab:test2}.

\begin{table}[H]
	\caption{Raport z drugiego testu}
	\label{tab:test2}
	\centering
	\begin{tabular}{|c|p{0.6\textwidth}|}
		\hline
		Testowane wymaganie & \textit{Użytkownik będzie mógł zapisać analizowane drzewa do pliku csv, do pliku binarnego oraz do bitmapy.} \\ \hline
		Kroki testowe & \begin{enumerate} \item Z menu głównego aplikacji wybierz ścieżkę do dowolnego pliku z zserializowanym drzewem. \item Naciśnij przycisk \textit{Inspect tree}. \item Naciśnij przycisk \textit{Save to csv file}. \item Naciśnij przycisk \textit{Save to binary file}. \item Naciśnij przycisk \textit{Save to bitmap file}. \item Sprawdź, czy bitmapa wygenerowana w kroku 5 odpowiada drzewu z pliku początkowego. \item Z menu głównego aplikacji wybierz ścieżkę plików wygenerowanych w kroku 3 i 4, żeby sprawdzić, czy zapisane drzewa wizualizowane są tak samo jak w początkowym pliku. \end{enumerate} \\ \hline
		Wynik & Pozytywny. \\ \hline
	\end{tabular}
\end{table}

Trzeci z przeprowadzonych testów dotyczy funkcjonalności modułu \textit{Wizualizacja}, a wynik opisany został w tabeli \ref{tab:test3}.

\begin{table}[H]
	\caption{Raport z trzeciego testu}
	\label{tab:test3}
	\centering
	\begin{tabular}{|c|p{0.6\textwidth}|}
		\hline
		Testowane wymaganie & \textit{Użytkownik będzie mógł wyświetlić informacje związane z wybranym węzłem drzewa, a także przybliżać i oddalać cały graf.} \\ \hline
		Kroki testowe & \begin{enumerate} \item Z menu głównego aplikacji wybierz ścieżkę do dowolnego pliku z drzewem. \item Naciśnij przycisk \textit{Inspect tree}. \item Przy użyciu prawego przycisku myszki chwyć za obszar rysowania i poruszaj się po wizualizacji. \item Używając kółka myszki, przybliż i oddal wizualizowane drzewo. \item Kliknij dowolny wierzchołek drzewa lewym przyciskiem myszki i sprawdź, czy panel z prawej strony wyświetla informacje związane z wybranym wierzchołkiem. \end{enumerate} \\ \hline
		Wynik & Pozytywny. \\ \hline
	\end{tabular}
\end{table}

Czwarty z przeprowadzonych testów dotyczy wydajności modułu \textit{Wizualizacja}, a wynik opisany został w tabeli \ref{tab:test4}.

\begin{table}[H]
	\caption{Raport z czwartego testu}
	\label{tab:test4}
	\centering
	\begin{tabular}{|c|p{0.6\textwidth}|}
		\hline
		Testowane wymaganie & \textit{Dla drzew do 100 000 wierzchołków wizualizacja nie powinna zajmować więcej niż 3s.} \\ \hline
		Kroki testowe & \begin{enumerate} \item Z menu głównego aplikacji wybierz ścieżkę do pliku \textit{tree\textunderscore 100k.csv}. \item Naciśnij przycisk \textit{Inspect tree}. \end{enumerate} \\ \hline
		Wynik & Pozytywny – deserializacja, ulepszony algorytm Walkera i wyświetlenie drzewa z pliku zajęło 2.802s. \\ \hline
	\end{tabular}
\end{table}

Piąty z przeprowadzonych testów dotyczy wydajności modułu \textit{Wizualizacja}, a wynik opisany został w tabeli \ref{tab:test5}.

\begin{table}[H]
	\caption{Raport z piątego testu}
	\label{tab:test5}
	\centering
	\begin{tabular}{|c|p{0.6\textwidth}|}
		\hline
		Testowane wymaganie & \textit{Dla drzew do 250 000 wierzchołków wizualizacja nie powinna zajmować więcej niż 5s.} \\ \hline
		Kroki testowe & \begin{enumerate} \item Z menu głównego aplikacji wybierz ścieżkę do pliku \textit{tree\textunderscore 250k.csv}. \item Naciśnij przycisk \textit{Inspect tree}. \end{enumerate} \\ \hline
		Wynik & Pozytywny – deserializacja, ulepszony algorytm Walkera i wyświetlenie drzewa z pliku zajęło 4.626s. \\ \hline
	\end{tabular}
\end{table}

Wszystkie testy akceptacyjne zakończyły się pozytywnie, a więc wymagania zostały spełnione.
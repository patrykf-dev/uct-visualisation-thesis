\documentclass[uct_visualisation_thesis.tex]{subfiles}

\section{Uzyskane efekty}
\section{Kontynuacja pracy}
\begin{enumerate}
	\item Dodanie kolejnych gier
	\item Przepisanie modułów odpowiedzialnych za obliczenia do C/C++
	\item Usprawnienie i zrównoleglenie symulacji szachów
	\item Dodanie księgi otwarć dla szachów
	\item Lepsza ewaluacja wartości figur w szachach
	\item Cache’owanie drzew w sekwencji
\end{enumerate}
\section{Wydajność}
\section{Testy akceptacyjne}
Testy akceptacyjne zostały przeprowadzone w celu sprawdzenia, czy aplikacja spełnia założenia opisane w dokumentacji wymagań projektu. Test \ref{tab:test1} konfrontuje założenia modułu \textit{Gry}, test \ref{tab:test2} – modułu \textit{Serializacja}, a pozostałe testy weryfikują założenia modułu \textit{Wizualizacja}.\\

Testy akceptacyjne zostały wykonane na komputerze:

\begin{itemize}
	\item z zainstalowanym systemem operacyjnym \textit{Windows 10 Education N},
	\item z zainstalowanym interpreterem języka \textit{Python 3.7.2} i biblioteką \textit{PyQt5},
	\item wyposażonym w procesor \textit{Intel Core i7-8700k @3.70 GHz},
	\item wyposażonym w kartę graficzną \textit{NVIDIA GeForce GTX 1060 6GB},
	\item wyposażonym w 32GB pamięci RAM.
\end{itemize}

Pierwszy z przeprowadzonych testów dotyczy funkcjonalności modułu \textit{Gry}, a wynik opisany został w tabeli \ref{tab:test1}.

\begin{table}[H]
	\caption{Raport z pierwszego testu}
	\label{tab:test1}
	\centering
	\begin{tabular}{|c|p{0.6\textwidth}|}
		\hline
		Testowane wymaganie & \textit{Użytkownik będzie mógł wybrać jedną z dwóch przykładowych gier, a do wyboru będzie miał trzy tryby rozgrywki.} \\ \hline
		Kroki testowe & \begin{enumerate} \item Z menu głównego aplikacji wybierz opcję \textit{Chess}. \item Z menu głównego aplikacji wybierz opcję \textit{Player vs player} i sprawdź tryb rozgrywki dla dwóch graczy. \item Z menu głównego aplikacji wybierz opcję \textit{Player vs PC} dla różnych ustawień algorytmu UCT. \item Z menu głównego aplikacji wybierz opcję \textit{PC vs PC} dla różnych ustawień algorytmu UCT. \item Z menu głównego aplikacji wybierz \textit{Mancala} i powtórz kroki 2–5. \end{enumerate} \\ \hline
		Wynik & Pozytywny. \\ \hline
	\end{tabular}
\end{table}

Drugi z przeprowadzonych testów dotyczy funkcjonalności modułu \textit{Serializacja}, a wynik opisany został w tabeli \ref{tab:test2}.

\begin{table}[H]
	\caption{Raport z drugiego testu}
	\label{tab:test2}
	\centering
	\begin{tabular}{|c|p{0.6\textwidth}|}
		\hline
		Testowane wymaganie & \textit{Użytkownik będzie mógł zapisać analizowane drzewa do pliku csv, do pliku binarnego oraz do bitmapy.} \\ \hline
		Kroki testowe & \begin{enumerate} \item Z menu głównego aplikacji wybierz ścieżkę do dowolnego pliku z zserializowanym drzewem. \item Naciśnij przycisk \textit{Inspect tree}. \item Naciśnij przycisk \textit{Save to csv file}. \item Naciśnij przycisk \textit{Save to binary file}. \item Naciśnij przycisk \textit{Save to bitmap file}. \item Sprawdź, czy bitmapa wygenerowana w kroku 5 odpowiada drzewu z pliku początkowego. \item Z menu głównego aplikacji wybierz ścieżkę plików wygenerowanych w kroku 3 i 4, żeby sprawdzić, czy zapisane drzewa wizualizowane są tak samo jak w początkowym pliku. \end{enumerate} \\ \hline
		Wynik & Pozytywny. \\ \hline
	\end{tabular}
\end{table}

Trzeci z przeprowadzonych testów dotyczy funkcjonalności modułu \textit{Wizualizacja}, a wynik opisany został w tabeli \ref{tab:test3}.

\begin{table}[H]
	\caption{Raport z trzeciego testu}
	\label{tab:test3}
	\centering
	\begin{tabular}{|c|p{0.6\textwidth}|}
		\hline
		Testowane wymaganie & \textit{Użytkownik będzie mógł wyświetlić informacje związane z wybranym węzłem drzewa, a także przybliżać i oddalać cały graf.} \\ \hline
		Kroki testowe & \begin{enumerate} \item Z menu głównego aplikacji wybierz ścieżkę do dowolnego pliku z drzewem. \item Naciśnij przycisk \textit{Inspect tree}. \item Przy użyciu prawego przycisku myszki chwyć za obszar rysowania i poruszaj się po wizualizacji. \item Używając kółka myszki, przybliż i oddal wizualizowane drzewo. \item Kliknij dowolny wierzchołek drzewa lewym przyciskiem myszki i sprawdź, czy panel z prawej strony wyświetla informacje związane z wybranym wierzchołkiem. \end{enumerate} \\ \hline
		Wynik & Pozytywny. \\ \hline
	\end{tabular}
\end{table}

Czwarty z przeprowadzonych testów dotyczy wydajności modułu \textit{Wizualizacja}, a wynik opisany został w tabeli \ref{tab:test4}.

\begin{table}[H]
	\caption{Raport z czwartego testu}
	\label{tab:test4}
	\centering
	\begin{tabular}{|c|p{0.6\textwidth}|}
		\hline
		Testowane wymaganie & \textit{Dla drzew do 100 000 wierzchołków wizualizacja nie powinna zajmować więcej niż 3s.} \\ \hline
		Kroki testowe & \begin{enumerate} \item Z menu głównego aplikacji wybierz ścieżkę do pliku \textit{tree\textunderscore 100k.csv}. \item Naciśnij przycisk \textit{Inspect tree}. \end{enumerate} \\ \hline
		Wynik & Pozytywny – deserializacja, ulepszony algorytm Walkera i wyświetlenie drzewa z pliku zajęło 2.802s. \\ \hline
	\end{tabular}
\end{table}

Piąty z przeprowadzonych testów dotyczy wydajności modułu \textit{Wizualizacja}, a wynik opisany został w tabeli \ref{tab:test5}.

\begin{table}[H]
	\caption{Raport z piątego testu}
	\label{tab:test5}
	\centering
	\begin{tabular}{|c|p{0.6\textwidth}|}
		\hline
		Testowane wymaganie & \textit{Dla drzew do 250 000 wierzchołków wizualizacja nie powinna zajmować więcej niż 5s.} \\ \hline
		Kroki testowe & \begin{enumerate} \item Z menu głównego aplikacji wybierz ścieżkę do pliku \textit{tree\textunderscore 250k.csv}. \item Naciśnij przycisk \textit{Inspect tree}. \end{enumerate} \\ \hline
		Wynik & Pozytywny – deserializacja, ulepszony algorytm Walkera i wyświetlenie drzewa z pliku zajęło 4.626s. \\ \hline
	\end{tabular}
\end{table}

Wszystkie testy akceptacyjne zakończyły się pozytywnie, a więc wymagania zostały spełnione.